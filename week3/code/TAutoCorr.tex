\documentclass[12pt]{article}
\usepackage{graphicx}
\usepackage{float}
\usepackage[left=2cm,right=2cm]{geometry}
\usepackage[font=scriptsize]{caption}
\graphicspath{{../results/}}

\begin{document}

\section*{\large \centering Are temperatures of one year significantly correlated with the next year 
(successive years), across years in a given location?}


The observed correlation coefficient between years and temperature(degree celsius) of Florida was
0.326. (Figure \ref{fig:mesh1}).

After random sampling our temperature data by \textit{N=10000} and recalculating the correlation coefficient,
our observed correlation coefficient was within the distribution of our random correlation
coefficients, but it was different to the mean (p=$4\times10^{-4}$) (Figure \ref{fig:mesh2}).

This suggest that there is a 
statistically significant moderate correlation between measured temperatures in adjacent years. This
allows us to conclude that there is significant auto-correlation in the data. As a result, we are
not able to draw conclusions about this data using statistical techniques that assume the data is
independent and identically distributed. Instead, we would need to use other techniques that can
handle dependence within the data.

In conclusion, there is a significant correlation between temperatures in successive years in a
given location, namely Florida.


\begin{figure}[H]
  \begin{minipage}{.45\textwidth}
        \includegraphics[width=.8\linewidth]{graph.pdf}
        \centering
        \caption{This graph demonstrates the annual temperatures from Key West in Florida, USA for the 20th century}
        \label{fig:mesh1}
  \end{minipage}%
  \begin{minipage}{.5\textwidth}
        \includegraphics[scale=.35]{graph2.pdf}
        \centering
        \caption{The distribution of random correlation coefficent calculated from reshuffling the temperatures. The observed correlation coefficent is represented by the trendline.}
        \label{fig:mesh2}
  \end{minipage}
\end{figure}

\end{document}

